% http://www.ctan.org/tex-archive/macros/latex/contrib/beamer/examples
% http://latex.artikel-namsu.de/english/beamer-examples.html

%\documentclass{beamer}
\documentclass[usenames,dvipsnames,12pt,compress]{beamer}
\setbeamertemplate{navigation symbols}{}
\usepackage[utf8]{inputenc}
\usepackage{amsmath}
\usepackage{amssymb}
\usepackage{bm}
\usepackage{fancybox, graphicx}
\usepackage{listings}
\usepackage{tikz} % Diagrams
\usetikzlibrary{positioning}
\usepackage{color}
\usepackage{textcomp} % See https://tex.stackexchange.com/questions/145416/how-to-have-straight-single-quotes-in-lstlistings
%\usepackage[font=small,labelfont=bf]{caption} % Required for specifying captions to tables and figures. From https://tex.stackexchange.com/questions/238636
\usepackage[absolute,overlay]{textpos}
\usepackage[T1,T2A]{fontenc} % See https://tex.stackexchange.com/questions/135118
\usepackage[russian,english]{babel}
\usepackage{url}


\lstset{language=bash,upquote=true} % Format listings as appropriate for bash. Inexplicably we get problems if the language is set as part of the \begin{lstlisting} command.

% https://tex.stackexchange.com/questions/36030/how-to-make-a-single-word-look-as-some-code
\definecolor{light-gray}{gray}{0.95}
\newcommand{\code}[1]{\colorbox{light-gray}{\texttt{#1}}}

\newcommand{\mentiurl}[0]{{\url{www.menti.com}}}
\newcommand{\menticode}[0]{{14 11 05 9}}
\newcommand{\mentiinvitation}[0]{Go to \mentiurl{} (code \menticode{}) and choose one possibility:\\}
\newcommand{\correctanswer}[1]{\textcolor{blue}{{#1} \checkmark}}

% Parameters: file name, graphics options (e.g. `scale=0.3'), whom to credit, x pos for credit, y pos for credit.
\newcommand{\imageandcredit}[5]{
  {
    \begin{tikzpicture}
      \draw (-4, 0) node[inner sep=0] {\includegraphics[{#2}]{{#1}}};
      \draw ({#4},{#5}) node[white,fill=black,font=\tiny] {Credit: {#3}};
    \end{tikzpicture}
  }
}

% Parameters: file name, graphics options (e.g. `scale=0.3'), background colour, whom to credit, x pos for credit, y pos for credit.
\newcommand{\framewithimageandcredit}[6]{
{
  \setbeamercolor{background canvas}{bg={#3}}
  \begin{frame}{}
    \begin{center}
      \imageandcredit{#1}{#2}{#4}{#5}{#6}
    \end{center}
  \end{frame}
 }
}


%Parameters: Item text, file name, graphics options, whom to credit, x pos for credit, y pos for credit.
\newcommand{\itemandimageandcredit}[6]{
  \begin{columns}
  \column{0.04\linewidth}
  \column{0.41\linewidth}
  \item{#1}
  \column{0.55\linewidth}
  \imageandcredit{#2}{#3}{#4}{#5}{#6}
  \end{columns}
}


%\usetheme{boxes}
%\usecolortheme{beaver}


\title{The Constantly Changing Hubble Constant}
\author{Lorne Whiteway \\ lorne.whiteway@star.ucl.ac.uk}
\institute{Astrophysics Group \\ Department of Physics and Astronomy \\ University College London}
\date{Presentation to the Mid Kent Astronomical Society \\ 12 November 2021 \\ You are invited to go to \alert{\mentiurl{}} and enter code \menticode{}.}

\begin{document}

\frame{\titlepage}


\begin{frame}{The Universe is expanding!}
  \begin{block}{}
  \begin{itemize}
  \item{But what does this actually mean?}
  \item{How do we know it is expanding?}
  \item{Why is it expanding?}
  \item{How fast is it expanding?}
  \item{Are cosmologists completely realistic about the uncertainties in their results?}
  \end{itemize}
  \end{block}
\end{frame}


\begin{frame}{How do we know?}
  \begin{columns}
    \column{0.4\linewidth}
    \begin{itemize}
    \item{Everywhere we look, distant galaxies are receding; more distant galaxies are receding faster.}
    \item{So either we are at the centre of a cosmic conspiracy, or all the space between all the galaxies is expanding.}
    \end{itemize}
    \column{0.6\linewidth}
    \centering
    \imageandcredit{720px-Hubble_ultra_deep_field_high_rez_edit1.jpg}{height=7cm}{NASA Ultra Deep Field}{-5.8}{-3.2}
  \end{columns}
\end{frame}


\begin{frame}{Is the solar system expanding? Are \textit{we} expanding?}
  \begin{block}{}
    \mentiinvitation{}
    \begin{enumerate}
      \item{Yes, a lot}
      \item{Yes, but only a tiny amount}
      \item{No}
    \end{enumerate}
  \end{block}
\end{frame}


\begin{frame}{Is the solar system expanding? Are \textit{we} expanding?}
  \begin{block}{}
    \mentiinvitation{}
    \begin{enumerate}
      \item{Yes, a lot}
      \item{Yes, but only a tiny amount}
      \item{\correctanswer{No}}
    \end{enumerate}
  \end{block}
\end{frame}


\begin{frame}{Is the solar system expanding? Are \textit{we} expanding?}
  \begin{block}{}
    \begin{itemize}
      \item{Other forces - molecular forces between the molecules in your body, and gravitational forces between the Sun and the planets - are far more than strong enough to overcome the effect of cosmic expansion.}
      \bigskip
      \itemandimageandcredit{Gravity is even strong enough to keep the Andromeda Galaxy from receding from us.}{Andromeda_Galaxy_560mm_FL.jpg}{height=2.5cm}{David Dayag}{-3.25}{-1.48}
      \bigskip
	\item{It's only the furthest objects - where gravity becomes negligible - that recede.}
    \end{itemize}
  \end{block}
\end{frame}


\begin{frame}{What does \textit{recession velocity} actually mean?}
  \begin{block}{}
  \begin{itemize}
  \item{We say `distant galaxies are moving away from us'. This is informal language.}
  \item{They aren't really moving, they just appear to be - because the intervening space is expanding.}
  \bigskip
  \item{Sometimes this makes a difference - for example, the recession velocity can exceed the speed of light.}
  \end{itemize}
  \end{block}
\end{frame}


\begin{frame}{Which `Ed' first had the idea that the Universe is expanding?}
  \begin{block}{}
    \mentiinvitation{}
    \begin{enumerate}
      \item{Edmond Halley}
      \item{Edwin Hubble}
      \item{Edgar Allan Poe}
    \end{enumerate}
  \end{block}
\end{frame}


\begin{frame}{Which `Ed' first had the idea that the Universe is expanding?}
  \begin{block}{}
    \mentiinvitation{}
    \begin{enumerate}
      \item{Edmond Halley}
      \item{Edwin Hubble}
      \item{\correctanswer{Edgar Allan Poe}}
    \end{enumerate}
  \end{block}
\end{frame}


\begin{frame}{History}
  \begin{block}{}
  \begin{itemize}
  \itemandimageandcredit{In 1848 Edgar Allan Poe published \textit{Eureka}, which included a description of expanding space.}{512px-Edgar_Allan_Poe,_circa_1849,_restored,_squared_off.jpg}{height=4cm}{Public domain}{-4}{-2.2}
  \item{Expansion is not obvious without large telescopes and so isn't usually part of pre-modern cosmologies. Full understanding only came in the 20th century.}
  \end{itemize}
  \end{block}
\end{frame}


\begin{frame}{So how fast is the expansion?}
  \begin{block}{}
  \begin{itemize}
  \item{For every additional distance of one megaparsec, there's an additional recession velocity of about 70 kilometers per second.}
  \item{So the expansion speed is about 70 kilometers per second per megaparsec.}
  \bigskip
  \item{One megaparsec is about three million light years. It's the typical distance between galaxies.}
  \item{70 kilometers per second is about 150,000 miles per hour.}
  \end{itemize}
  \end{block}
\end{frame}


\begin{frame}{So how fast is the expansion?}
    \begin{tikzpicture}
    \node at (4,1) {Start with a distance:};
    \draw [very thick] (0,0)--(10,0);
    \pause
    \node at (4,-2) {13.5 million years later it will be 1\% longer:};
    \draw [very thick] (0,-3)--(10.1,-3);
    \pause
    \node at (4,-5) {Continental drift is about six times faster...};
    \end{tikzpicture}
\end{frame}


\begin{frame}{$H_0$}
  \begin{block}{}
  \begin{itemize}
  \item{The current expansion rate is called the \textit{Hubble constant} or \textit{Hubble parameter} and is denoted `$H_0$'.}
  \bigskip
  \itemandimageandcredit{The `$H$' commemorates Edwin Hubble (1889-1953), who was one of the first to measure it.}{Studio_portrait_photograph_of_Edwin_Powell_Hubble.jpg}{height=3cm}{Johan Hagemeyer}{-4}{-1.3}
  \bigskip
  \item{The `$0$' refers to today. The expansion rate was different in the distant past.}
  \end{itemize}
  \end{block}
\end{frame}
 
 
\begin{frame}{Why does the Universe expand?}
  \begin{block}{}
  \begin{itemize}
  \item{Science is not so good with `why?' questions...}
  \bigskip
  \item{There's an \textit{initial condition}: the Universe started expanding at the Big Bang.}
  \bigskip
  \item{The later behaviour of the expansion (does it slow down? speed up?) then depends, essentially via gravity, on \textit{what's in the Universe}.}
  \end{itemize}
  \end{block}
\end{frame}
  
  
\begin{frame}{Why does gravity play a role?}
  \begin{block}{}
  \begin{itemize}
  \itemandimageandcredit{\textit{General relativity}, our modern theory of gravity, is due to Einstein (1916).}{08608_einstein_1916.jpg}{height=6cm}{Paul Ehrenfest}{-5.03}{-3.2}
  \end{itemize}
  \end{block}
\end{frame}


\begin{frame}{Why does gravity play a role?}
  \begin{block}{}
  \begin{itemize}
  \item{Remember \textit{mass} and \textit{energy} are the same ($E=mc^2$).}
  \bigskip
  \itemandimageandcredit{Mass/Energy \textit{bends} spacetime, essentially changing distances and angles.}{800px-Spacetime_lattice_analogy.svg.png}{height=2cm}{Mysid}{-4}{-1.3}
  \end{itemize}
  \end{block}
\end{frame}


\begin{frame}{Why does gravity play a role?}
  \begin{block}{}
  \begin{itemize}
  \item{This `changing of distances and angles' works locally; the distorted spacetime governs how objects move, and this leads e.g. to the apple falling from the tree.}
  \bigskip
  \pause
  \item{But it also works on the Universe as a whole - mass/energy can cause distances to change \textit{everywhere} in the Universe - and in particular can lead to increasing distances everywhere. This is the expansion that we see.}
  \end{itemize}
  \end{block}
\end{frame}


\begin{frame}{Contents of Universe control expansion}
  \begin{block}{}
  \begin{itemize}
  \itemandimageandcredit{It was Alexander Friedmann (\begin{otherlanguage*}{russian}Алекс\'{а}ндр Алекс\'{а}ндрович Фр\'{и}дман\end{otherlanguage*}) (1888-1925) who first realised this (1922).}{Aleksandr_Fridman.png}{height=6cm}{Public domain}{-4}{-3.2}
  \end{itemize}
  \end{block}
\end{frame}


\begin{frame}{What if we go backwards in time?}
  \begin{block}{}
  \begin{itemize}
  %https://upload.wikimedia.org/wikipedia/commons/thumb/5/51/Georges_Lema%C3%AEtre_1930s.jpg/512px-Georges_Lema%C3%AEtre_1930s.jpg
  \itemandimageandcredit{George Lama\^itre (1894-1966) realised that if the Universe was expanding then it must, at an earlier stage, have been very small; he thereby invented the idea of the `Big Bang'.}{512px-Georges_Lemaitre_1930s.jpg}{height=6cm}{Public domain}{-4}{-3.2}
  \end{itemize}
  \end{block}
\end{frame}


\begin{frame}{How do we measure the expansion rate?}
  \begin{block}{}
  \begin{itemize}
  \item{In theory it's easy: find a distant galaxy, measure its recession velocity and its distance, and take the ratio.}
  \bigskip
  \item{Example: a galaxy is receding at 1600 kilometers per second and is 20 megaparsecs away; \linebreak \bigskip then $H_0$ is 80 kilometers per second per megaparsec.}
  \end{itemize}
  \end{block}
\end{frame}


\begin{frame}{For a distant galaxy, which is harder to measure?}
  \begin{block}{}
    \mentiinvitation{}
    \begin{enumerate}
      \item{Recession velocity}
      \item{Distance}
      \item{They are about the same difficulty}
    \end{enumerate}
  \end{block}
\end{frame}


\begin{frame}{For a distant galaxy, which is harder to measure?}
  \begin{block}{}
    \mentiinvitation{}
    \begin{enumerate}
      \item{Recession velocity}
      \item{\correctanswer{Distance}}
      \item{They are about the same difficulty}
    \end{enumerate}
  \end{block}
\end{frame}


\begin{frame}{Can use \textit{redshift} to measure recession velocity}
  \begin{block}{}
  \begin{itemize}
  \item{Light from distant galaxies gets streeeetched by the expansion; this makes it turn redder.}
  \bigskip
  \itemandimageandcredit{It's fairly easy to measure the amount of red-shifting, as spectral lines are a convenient reference point. The redshift then immediately gives the velocity.}{256px-Redshift.svg.png}{height=4cm}{Georg Wiora}{-4}{-2}
  \end{itemize}
  \end{block}
\end{frame}


\begin{frame}{V M Slipher}
  \begin{block}{}
  \begin{itemize}
  \itemandimageandcredit{The first redshifts for galaxies (known then as nebulae) were made in 1912 by Vesto Slipher (1875 - 1969) at the Lowell Observatory in Flagstaff Arizona.}{V.M._Slipher.png}{height=6cm}{Lowell Archives}{-4}{-3}
  \end{itemize}
  \end{block}
\end{frame}



%%%%%%%%%%%%%%% Start of end material %%%%%%%%%%%%%%%



\begin{frame}{Where to find the document}
  \begin{block}{}
  \begin{itemize}
  \item{You can find the presentation at \alert{\url{https://tinyurl.com/bycke8v6}}}
  \end{itemize}
  \end{block}
\end{frame}


\begin{frame}{Image credits}
  \begin{block}{}
  \begin{enumerate}
  \tiny {
  \item{Hubble Deep Field: Source: NASA. From \url{https://en.wikipedia.org/wiki/File:Hubble_ultra_deep_field_high_rez_edit1.jpg}}
  \item{M31: Source: David Dayag. Licensed under the Creative Commons Attribution-Share Alike 4.0 International license \url{https://creativecommons.org/licenses/by-sa/4.0/deed.en}. From \url{https://upload.wikimedia.org/wikipedia/commons/thumb/8/8c/Andromeda_Galaxy_560mm_FL.jpg/1024px-Andromeda_Galaxy_560mm_FL.jpg}.}
  \item{Poe: Source: Public domain. From \url{https://en.wikipedia.org/wiki/Edgar_Allan_Poe\#/media/File:Edgar_Allan_Poe,_circa_1849,_restored,_squared_off.jpg}}
  \item{Hubble: Source: Public domain (by Johan Hagemeyer (1884-1962)). From \url{https://commons.wikimedia.org/wiki/File:Studio_portrait_photograph_of_Edwin_Powell_Hubble.JPG}}
  \item{Einstein: Source: Public domain. From \url{https://upload.wikimedia.org/wikipedia/commons/thumb/c/c3/08608_einstein_1916.jpg/512px-08608_einstein_1916.jpg}}
  \item{Curved spacetime: Source: Mysid. Licensed under the Creative Commons Attribution-Share Alike 3.0 Unported license \url{https://creativecommons.org/licenses/by-sa/3.0/deed.en}. From \url{https://commons.wikimedia.org/wiki/File:Spacetime_lattice_analogy.svg}}
  \item{Friedmann: Source: Public domain. From \url{https://upload.wikimedia.org/wikipedia/commons/6/62/Aleksandr_Fridman.png}}
  \item{Lema\^itre: Source: Public domain. From \url{https://upload.wikimedia.org/wikipedia/commons/thumb/5/51/Georges_Lema\%C3\%AEtre_1930s.jpg/512px-Georges_Lema\%C3\%AEtre_1930s.jpg}}  
  \item{Redshift: Source: Georg Wiora. Llicensed under the Creative Commons Attribution-Share Alike 2.5 Generic license \url{https://creativecommons.org/licenses/by-sa/2.5/deed.en}. From \url{https://commons.wikimedia.org/wiki/File:Redshift.svg}}
  \item{Slipher: Source: Lowell Archive. Licensed under the Creative Commons Attribution-Share Alike 4.0 International license \url{https://creativecommons.org/licenses/by-sa/4.0/deed.en}. From \url{https://upload.wikimedia.org/wikipedia/commons/a/a7/V.M._Slipher.gif}, converted to png}  
  } % end footnotesize
  \end{enumerate}
  \end{block}
\end{frame}

\end{document}
